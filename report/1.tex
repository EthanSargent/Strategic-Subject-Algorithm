The Strategic Subject Algorithm (SSA) was developed at the Illinois Institute of Technology in 2013 and ranked hundreds of thousands of Chicagoans based on their estimated likelihood of involvement in a shooting, as either as a victim or a perpetrator \cite{nyt}.\\\\
After a FOIA request by the Chicago Sun-Times \cite{upturn}, the City of Chicago released the 2017 version of the Strategic Subject List (SSL), which is available at \cite{data}. Each row of the list represents a subject. Fields include each subject's SSL score, the 8 fields used as inputs to the SSA in computing this score, and 44 other fields including such sensitive information as race, gender, and geographical location that the algorithm does not incorporate. The details of the algorithm itself are not public, nor are the criteria for inclusion on the list - per \cite{upturn}, "127,513 individuals on the list have never been arrested or shot." These criteria and the algorithm itself are periodically updated - only 102 of the 398,684 subjects on the most recent list have no recent arrest date.\\\\
The New York Times \cite{nyt} and Upturn \cite{upturn} conducted highly successful regression analyses of SSL score, although they did not publicize the details of their models. What follows is an additional regression analysis which is therefore not intended as original research, but is instead offered in the spirit of exposition on the Strategic Subject Algorithm. We begin with exploratory data analysis of the SSL and its 53 fields. Next, we present several accurate models of SSL score as a function of its 8 predictor variables. Finally, we attend to the obfuscated predictor variable {\tt TREND IN CRIMINAL ACTIVITY}. As with SSL score, the mechanism by which this variable is computed is unknown, so we construct a model which predicts this variable.\\\\
In a Chicago Police Department directive \cite{directive}, the SSL was announced to have been discontinued effective January 9th, 2019. The SSL and SSA were replaced by the Subject Assessment and Information Dashboard (SAID)
 and the Crime and Victimization Risk Model (CVRM), which model is described in detail at \cite{factsheet}. The CVRM takes as inputs 6 of the 8 inputs to the SSA, was also designed by IIT, and appears to have the same function of predicting subjects' short-term likelihood of either committing or being a victim of a shooting. Unlike the SSA, the CVRM represents the subject list as an undirected graph, reflecting IIT's understanding that "[i]n prior research on violent crime, [correlations] have been extensively demonstrated to exist among individuals who are close to one another in a graph defined by co-arrests." To my knowledge no other information about the SAID or the CVRM is available to the public.
%\newline \newline \par ETHAN SARGENT

%Add Picture in and Uncomment
%\par \includegraphics[width=4cm,height=2cm]{Final/1_Writing/sign.png}
%\par \today   
